%!TEX root = ../thesis.tex
\section{INTRODUCTION}

\subsection{Background}
% In recent years, there have been many papers describing ...

\subsection{Objectives and Delimitations}
The objective of this thesis is to design a software for transport accessibility analysis
which can used for by people without special background in the field. Another important
requirement is simplicity of the distribution. Thus, software had to be implemented
as a web service which can accessed by any modern browser. The performance of the interface
is crucial, hence the user would not be irritated or confused. The delays related to the processing,
rendering and data transfer should be minimized or illuminated completely.

It is expected that the program will be useful for various groups of actors which
were divided in 5 categories such as state institution, logistics, real estate, science
and citizens. The whole list of the potential actors is presented on Table~\ref{tab:actors}.

\begin{table}[ht]
  \renewcommand{\arraystretch}{1.5}
  \centering
  \footnotesize
  \begin{tabular}{p{4cm} p{2.5cm} p{2.5cm} p{1.5cm} p{2.2cm}}
    \hline
    \textbf{State Institutions} & \textbf{Logistics} &\textbf{Real Estate}
    & \textbf{Science} & \textbf{Citizens} \\
    \hline

    Department of Complete Overhaul of Moscow &
    Trading companies &
    Realtors &
    Mapping &
    Tourists \\

    %%

    Department of Culture of Moscow &
    Logistics companies &
    Post officers &
    Ecology &
    Travelers  \\

    %%

    Department for Housing, Utilities and Amenities &
    &
    Collector Logistics &
    Sociology &
    Guest workers \\

    %%

    Department for Transport and Road Infrastructure Development
    &
    &
    Political science &
    \\

    %%

    Department for Urban Development and Construction &
    &
    &
    &
    \\

    \hline
  \end{tabular}

  \caption{Potential actors.}
  \label{tab:actors}
\end{table}

In contrast with other navigational services like Google Maps and Yandex Maps the software is
focused on the ability to evaluate current state of the transport accessibility in different regions
of Moscow. For the Department for Transport and Road Infrastructure Development the service could be
useful on the stage of the planning new infrastructures like roads and bridges. For the Department for
Urban Development and Construction the application may provide an understanding of the
accessibility level in certain location. In the process of building new store or cinema service
would also help visualize how much time does it take to get to the location.

It is important to note that this work does not contain any analysis of the transport accessibility
in Moscow. The main focus of the work is the development of the software for analysis. It is also
crucial that software the same software could be used for analysis of any other city, the only
thing that will change is the data processing step.

\subsection{Structure of the Report}

The thesis is divided into 7 main sections. The first section is an introduction where the
background and objectives are discussed. Next section is devoted to the overview of the tools which
are currently available for geographical analysis, libraries for mapping in the browser and
description of the data formats used for storing geographical data. After that, comes software
requirements section where all of the mandatory features are listed. Then, the architecture and
implementation are described. Next, the development environment and deployment process is outlined on
integration section. In the last section final results are presented. There is also an appendix
containing screen shots of the different states of the application with the description.
