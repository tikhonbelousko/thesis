%!TEX root = ../thesis.tex
\section{OVERVIEW}

This section is devoted to the description of the available geo-spatial data analysis tools. Also,
modern web-based maps libraries are considered. Finally, data formats used for storing geographical
data are reviewed.

\subsection{GIS Tools for Accessibility Analysis}

% QGIS + CartoDB
There are

\subsection{Modern Web-based Maps}
In our application the choice of maps library should be made. In this section several
libraries for maps drawing are reviewed. The pluses and minuses of each of the
library are considered.

\begin{description}

  % Google
  \item[Google Maps Javascript API] -- on of the most popular libraries
    on the list~\cite{google:maps}. \\
    \underline{\smash{Pluses:}}
    \begin{itemize}
      \item Maps styling.
      \item Supports different localization options.
      \item The most detailed map across the world.
      \item StreetView.
      \item Support for mobile devices.
    \end{itemize}

    \underline{\smash{Minuses:}}
    \begin{itemize}
      \item Does not support vector tiles.
      \item Limited support for custom tiles.
      \item Free until exceeding 25,000 map loads per 24 hours for 90 consecutive days.
    \end{itemize}

  % Leaflet
  \item[Leaflet.js] -- lightweight mobile-friendly interactive maps
    library~\cite{leaflet}. \\
    \underline{\smash{Pluses:}}
    \begin{itemize}
      \item Small size -- only 33 KB of JavaScript code.
      \item Relatively simple API.
      \item Support for custom tile sources.
      \item Support for multilayer maps.
      \item Considerable amount of plug-ins.
      \item Free and open-source.
    \end{itemize}

    \underline{\smash{Minuses:}}
    \begin{itemize}
      \item Does not support vector tiles.
      \item Lack of Russian localization.
    \end{itemize}

  % Yandex Maps
  \item[Yandex Maps] -- best maps fidelity in Russia~\cite{yandex:maps}. \\
    \underline{\smash{Pluses:}}
    \begin{itemize}
      \item Highly detailed maps of Russia.
      \item Support of Russian localization.
      \item No restrictions for map views.
    \end{itemize}

    \underline{\smash{Minuses:}}
    \begin{itemize}
      \item Does not support vector tiles.
      \item Does not support map styles.
      \item Does not support custom tiles rendering.
      \item Proprietary.
    \end{itemize}

  % Mapbox GL JS
  \item[Mapbox GL JS] -- only library with whole vector tiles support~\cite{gh:mapboxgljs}. \\
    \underline{\smash{Pluses:}}
    \begin{itemize}
      \item Supports vector tiles.
      \item Supports real-time maps styles.
      \item Supports custom tile source.
      \item Based on Leaflet API.
      \item Open source.
    \end{itemize}

    \underline{\smash{Minuses:}}
    \begin{itemize}
      \item Standard tile source has limitations on views per month.
      \item Not fully detailed map of Russia.
      \item Lack of Russian localization.
    \end{itemize}
\end{description}

\subsection{Data Formats}

In the process of the development of transport accessibility analysis tool several
particular data formats were used. In this section utilized formats will be described.

\subsubsection{GeoJSON}

One of the most extensively used formats for encoding various types of geographical data is
GeoJSON which is actually a subset of JSON format~\cite{geojson:spec}. GeoJSON can store information
about such objects as Point, MultiPoint, Line, LineString, MultiLineString, Polygon. Geometric
objects which have some additional properties are stored as Feature objects. Moreover,
Features can be organized in feature collection. The example of the location which
represents center of the Moscow can be encoded as described on Listing~\ref{lst:geojson}.

\begin{lstlisting}[language=json, caption=GeoJSON data example.,
      label={lst:geojson}]
{
  "type": "Feature",
  "geometry": {
    "type": "Point",
    "coordinates": [37.61581, 55.74489]
  },
  "properties": {
    "name": "The Center of Moscow"
  }
}
\end{lstlisting}


The GeoJSON is text data format which results in larger size of the transfered data in comparison
with binary formats. Although, usually on the server it can be compressed using gzip~\cite{gzip}
which has native support in modern browsers. Compression results in dramatical size reduction up to
5 times in general case. However, it should be mentioned that archiving adds small overhead in
coding and decoding steps.

\subsubsection{Geobuf}

Geobuf is a compact data format for encoding geographical data. The format is based on Protocol
Buffers -- language-neutral mechanism developed by Google for serializing structured
data~\cite{protobuf}. Geobuf allows losslessly transformation of the GeoJSON data into protocol
buffers. There are several important advantages like faster compression in comparison to even native
JSON \texttt{parse} and \texttt{stringify} methods. Another valuable attribute of this format is
compact size, which is 8 times smaller than raw GeoJSON and 2 times smaller than GeoJSON after
applying gzip compression. The comparison of GeoJSON and Geobuf in terms size is presented on
Table~\ref{tab:geobuf}.

\begin{table}[ht]
  \renewcommand{\arraystretch}{1.5}
  \centering
  \begin{tabular}{l l l}
    \hline
    & \textbf{normal} & \textbf{gzipped} \\
    \hline
    us-zips.json & 101.85 MB & 26.67 MB \\
    % \hline
    us-zips.pbf & 12.24 MB & 10.48 MB \\
    % \hline
    idaho.json & 10.92 MB & 2.57 MB \\
    % \hline
    idaho.pbf & 1.37 MB & 1.17 MB \\
    \hline
  \end{tabular}

  \caption{Sample compression sizes~\cite{geobuf}.}
  \label{tab:geobuf}
\end{table}

\subsubsection{MBTiles}

Web maps may contain million of tiles, hence there is clearly a problem in storing and managing such
enormous amount of data. To overcome the problem of handling so many tiles data Mapbox team has
developed open-source data format for storing tiles in a single SQLite data base.

SQLite is claimed to be ideal for the purpose of storing tiles since it is available on all of the
platforms including mobile devices. Each \texttt{.sqlite} data base is self-contained and does not
require any special setup, which results in high portability.

Another great feature of the MBTiles format is the ability to effectively store duplicate tiles. As
an example the tile located in the middle of the ocean can be considered (see
Figure~\ref{pic:water-tile}). It is clear that there is considerable amount of tiles containing
solid blue color. Taking in account all of the zoom levels it can lead to millions of the
duplicates. However, MBTiles can reference thousands of tiles to the same image without
the need for loading all look-alike pictures.

Important to note that MBTiles can be used to store both image-based tiles encoded in PNG or
vector tiles encoded in Protocol Buffers.

\begin{figure}[htp]
  {\par\centering
  \includegraphics[width=0.20\textwidth]{water-tile}
  \par}
  \caption{Example of the tile duplicate.}
  \label{pic:water-tile}
\end{figure}






