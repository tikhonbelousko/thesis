%!TEX root = ../thesis.tex
\section{INTEGRATION}

One of the important aspects of the development process is the set of tools which were
used. In this section the particular methods and tools will be described, which
were applied on course of making this program. Another part will be devoted to
the deployment.

\subsection{Development Environment}

Right set of tools can significantly speed up development process by lowering probability of making
an error, provide extensions to the language which may help to structure code for higher
maintainability. In out project, tools may be split into two categories, namely client-side and
server-side utilities.

When JavaScript was used for simple interactions on the HTML page, the development of the client
code was quite straight-forward: create \texttt{.js} file, write some code, include the script to
the index.html. Unfortunately, when applications started to become more complex it was clear that
code had to be split into reusable modules. This problem was solved in ES2015 JavaScript standard
[REFERENCE] which introduced the concept of modules, however, now only small percent of the browsers
support new standard. To solve the problem with compatibility the Webpack was chosen. Webpack is a
module bundler which allows split the code into different files. Once the project would need to
built webpack is going through all of the imports and packing all modules in one solid bundle which
can be easily included into HTML page.

One of the core concepts of the webpack is loaders. Loader is a transformation
of the file. There are pre-installed loaders such as JavaScript loader for
bundling JavaScript files together. Loaders can be chained, allowing to use modern
JavaScript syntax that can be transformed to normal JavaScript. Webpack
works not only with JavaScript, but also with CSS so it becomes possible to include
CSS files in JavaScript and use plug-ins like ``autoprefixer'' to write cleaner styles.

It is a common problem when different project use different versions of the same
package. To overcome this issue on the server the pyenv was used. Pyenv allows
to create virtual environment where you can locally install all needed packages. Virtual
environments are completely isolated which illuminates the problem of the dependencies version
conflict.

\subsection{Deploying the Project}

To quickly run project in development environment the make tool was used. There is a Makefile
which runs all needed commands to start up three servers on different ports. However,
to run project on remote server four steps need to be performed:

\begin{enumerate}
  \item Install all dependencies.
  \item Build client code.
  \item Install proxy server such nginx [REFERENCE] and wire up tile server and API server ports with it.
  \item Setup nginx for serving statics (HTML, JavaScript and assets) on 80 port.
  \item Start the tile server and the API server.
\end{enumerate}